\documentclass[11pt]{article}
\usepackage[a4paper,margin=2cm]{geometry}
\usepackage{polski}
\usepackage[polish]{babel}
\usepackage[utf8]{inputenc}

% dodatkowe pakiety
\usepackage{enumerate}
\usepackage{longtable}
\usepackage{verbatim}

% figures
\usepackage{float}

\usepackage{url}

% linki
\usepackage[hyperref]{xcolor}
\usepackage{hyperref}
\definecolor{link-blue}{rgb}{0.15,0.15,0.4}
\hypersetup{
    colorlinks = true,
    linkcolor = link-blue,
    citecolor = link-blue,
    filecolor = link-blue,
    urlcolor  = link-blue
}

\title{\textbf{Big Data -- opracowanie}}
\author{Grzegorz Cichosz\\
    Michał Lenart\\
    Mateusz Rzeszutek\\
    Dariusz Świętek}
\date{}

% ustawia rodzaje kropek przy wyliczeniach (itemize)
\renewcommand{\labelitemi}{$\bullet$}
\renewcommand{\labelitemii}{$\circ$}
\renewcommand{\labelitemiii}{$\cdot$}
\renewcommand{\labelitemiv}{$\ast$}

\begin{document}

\maketitle

\begin{abstract}
    Big data oznacza dosłownie ,,duże dane''. Termin ten powstał stosunkowo niedawno i jest on używany wtedy, gdy mówi się o przetwarzaniu bardzo dużych ilości danych. Na zaistnienie tej dość specyficznej dziedziny danych miał duży wpływ rozwój Internetu -- ilość danych pochodzących chociażby z Twittera jets monstrualna.
\end{abstract}

%\tableofcontents
%\clearpage

%---------------------------------------------------------------------------
\section{Wstęp}
\label{sec:wstep}

% TODO
Termin ,,Big Data'' odnosi się do wielkich zbiorów danych, często liczących nawet petabajty danych. Takie ilości danych powstają np. w wyniku obserwacji teleskopem. Tak ogromnych zbiorów danych nie da się już analizować konwencjonalnymi metodami -- tradycyjne, relacyjne bazy danych nie radziły sobie z przechowywaniem i analizą aż tak wielu informacji. Na dodatek dane te często bywają pozbawione wyraźnej struktury (np. posty na Twitterze), co jeszcze bardziej pogarszało performance takich baz danych.

Obecnie do analizy tak dużych ilości danych uzywa się platform takich jak Hadoop \cite{apache:hadoop}, czy platformy Google. Są to platformy rozproszone, oparte na specjalnie zaprojektowanych systemach plików (np. HDFS -- Hadoop Distributed File System, opisany w \cite{shvachko10}), na których stoją bazy danych NoSQL (HBase, BigTable). Do przetwarzania danych powstał paradygmat MapReduce \cite{dean08}, który w prosty sposób przeprowadza rozproszone obliczenia na danych przechowywanych w systemie. Aby ułatwić korzystanie z tych systemów powstały takie nakładki jak Apache Pig, które umożliwiają wykonywanie zapytań i obliczeń za pomocą prostych języków, nieco przypominających SQL.

% section wstep (end)

\section{Rozproszone systemy plików}
\label{sec:rozproszone_systemy_plikow}

% TODO

% section rozproszone_systemy_plikow (end)

\section{Bazy danych NoSQL}
\label{sec:bazy_danych_nosql}

% TODO

% section bazy_danych_nosql (end)

\section{MapReduce}
\label{sec:mapreduce}

% TODO

% section mapreduce (end)

%---------------------------------------------------------------------------

\nocite{*}
\bibliographystyle{abbrv}
\bibliography{literatura}

\end{document}
