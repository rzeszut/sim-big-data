\documentclass[11pt]{article}
\usepackage[a4paper,margin=2cm]{geometry}
\usepackage{polski}
\usepackage[polish]{babel}
\usepackage[utf8]{inputenc}

% dodatkowe pakiety
\usepackage{enumerate}
\usepackage{longtable}
\usepackage{verbatim}

% figures
\usepackage{float}

\usepackage{url}

% linki
\usepackage[hyperref]{xcolor}
\usepackage{hyperref}
\definecolor{link-blue}{rgb}{0.15,0.15,0.4}
\hypersetup{
    colorlinks = true,
    linkcolor = link-blue,
    citecolor = link-blue,
    filecolor = link-blue,
    urlcolor  = link-blue
}

\title{\textbf{Big Data -- opracowanie}}
\author{Grzegorz Cichosz\\
    Michał Lenart\\
    Mateusz Rzeszutek\\
    Dariusz Świętek}
\date{}

% ustawia rodzaje kropek przy wyliczeniach (itemize)
\renewcommand{\labelitemi}{$\bullet$}
\renewcommand{\labelitemii}{$\circ$}
\renewcommand{\labelitemiii}{$\cdot$}
\renewcommand{\labelitemiv}{$\ast$}

\begin{document}

\maketitle

\begin{abstract}
    Big data oznacza dosłownie ,,duże dane''. Termin ten powstał stosunkowo niedawno i jest on używany wtedy, gdy mówi się o przetwarzaniu bardzo dużych ilości danych. Na zaistnienie tej dość specyficznej dziedziny danych miał duży wpływ rozwój Internetu -- ilość danych pochodzących chociażby z Twittera jets monstrualna.
\end{abstract}

\tableofcontents
\clearpage

\section{Wprowadzenie}
\label{sec:Wprowadzenie}

% TODO
Termin ,,Big Data'' odnosi się do wielkich zbiorów danych, często liczących nawet petabajty danych. Takie ilości danych powstają np. w wyniku obserwacji teleskopem. Tak ogromnych zbiorów danych nie da się już analizować konwencjonalnymi metodami -- tradycyjne, relacyjne bazy danych nie radziły sobie z przechowywaniem i analizą aż tak wielu informacji. Na dodatek dane te często bywają pozbawione wyraźnej struktury (np. posty na Twitterze), co jeszcze bardziej pogarszało performance takich baz danych.

Obecnie do analizy tak dużych ilości danych uzywa się platform takich jak Hadoop \cite{apache:hadoop}, czy platformy Google. Są to platformy rozproszone, oparte na specjalnie zaprojektowanych systemach plików (np. HDFS -- Hadoop Distributed File System, opisany w \cite{shvachko10}), na których stoją bazy danych NoSQL (HBase, BigTable). Do przetwarzania danych powstał paradygmat MapReduce \cite{dean08}, który w prosty sposób przeprowadza rozproszone obliczenia na danych przechowywanych w systemie. Aby ułatwić korzystanie z tych systemów powstały takie nakładki jak Apache Pig, które umożliwiają wykonywanie zapytań i obliczeń za pomocą prostych języków, nieco przypominających SQL.

% section wprowadzenie (end)

\section{Stan badań w zakresie Big Data}
\label{sec:stan_badan_w_zakresie_big_data}

% TODO

% section stan_badan_w_zakresie_big_data (end)

\section{Wykaz polskich instytucji naukowo-badawczych prowadzących badania w zakresie Big Data}
\label{sec:wykaz_polskich_instytucji_naukowo_badawczych_prowadz_cych_badania_w_zakresie_big_data}

% TODO

% section wykaz_polskich_instytucji_naukowo_badawczych_prowadz_cych_badania_w_zakresie_big_data (end)

\section{Polskie projekty badawcze z zakresu Big Data}
\label{sec:polskie_projekty_badawcze_z_zakresu_big_data}

% TODO

% section polskie_projekty_badawcze_z_zakresu_big_data (end)

\section{Analiza patentowa}
\label{sec:analiza_patentowa}

% TODO

% section analiza_patentowa (end)

\section{Analiza bibliograficzna}
\label{sec:analiza_bibliograficzna}

% TODO

% section analiza_bibliograficzna (end)

\section{Trendy}
\label{sec:trendy}

\subsection{Trend Big Data}
\label{sub:trend_big_data}
\begin{tikzpicture}
    \begin{axis}[
            xlabel=Rok,
            ylabel=Ilość artykułów,
            /pgf/number format/.cd,
            use comma,
            grid=major,
            1000 sep={}
        ]
        \addplot[color=blue, mark=o] coordinates {
            (1995, 57) 
            (1996, 54) 
            (1997, 71) 
            (1998, 86) 
            (1999, 64) 
            (2000, 68) 
            (2001, 80) 
            (2002, 68) 
            (2003, 90) 
            (2004, 86) 
            (2005, 117) 
            (2006, 114) 
            (2007, 123) 
            (2008, 139) 
            (2009, 161) 
            (2010, 132) 
            (2011, 130) 
            (2012, 176) 
            (2013, 169) 
        };
    \end{axis}
\end{tikzpicture}

% section trendy (end)

\section{Propozycje cenariuszy rozwoju technologicznego i rynku Big Data}
\label{sec:propozycje_cenariuszy_rozwoju_technologicznego_i_rynku_big_data}

% TODO

% section propozycje_cenariuszy_rozwoju_technologicznego_i_rynku_big_data (end)

\section{Programy edukacyjne, prace dyplomowe i doktorskie}
\label{sec:programy_edukacyjne_prace_dyplomowe_i_doktorskie}

% TODO

% section programy_edukacyjne_prace_dyplomowe_i_doktorskie (end)

\section{Zastosowania i potrzeby wdrożeniowe}
\label{sec:zastosowania_i_potrzeby_wdro_eniowe}

% TODO

% section zastosowania_i_potrzeby_wdro_eniowe (end)

\section{Nagrody, udział w towarzystwach naukowych, recenzje, konferencje, sympozja, komitety redakcyjne i pozostała działalność naukowa polskich zespołów w zakresie Big Data}
\label{sec:nagrody_udzia_w_towarzystwach_naukowych_recenzje_konferencje_sympozja_komitety_redakcyjne_i_pozosta_a_dzia_alno_naukowa_polskich_zespo_w_w_zakresie_big_data}

% TODO

% section nagrody_udzia_w_towarzystwach_naukowych_recenzje_konferencje_sympozja_komitety_redakcyjne_i_pozosta_a_dzia_alno_naukowa_polskich_zespo_w_w_zakresie_big_data (end)

\section{Proponowane problemy z zakresu Big Data, które mogłyby być badane metodą delficką}
\label{sec:proponowane_problemy_z_zakresu_big_data_kt_re_mog_yby_by_badane_metod_delfick_}

% TODO

% section proponowane_problemy_z_zakresu_big_data_kt_re_mog_yby_by_badane_metod_delfick_ (end)

\section{Podsumowanie}
\label{sec:podsumowanie}

% TODO

% section podsumowanie (end)


\nocite{*}
\bibliographystyle{abbrv}
\bibliography{literatura}

\end{document}
