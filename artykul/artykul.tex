%%%%%%%%%%%%%%%%%%%%%%% file template.tex %%%%%%%%%%%%%%%%%%%%%%%%%
%
% This is a general template file for the LaTeX package SVJour3
% for Springer journals.          Springer Heidelberg 2010/09/16
%
% Copy it to a new file with a new name and use it as the basis
% for your article. Delete % signs as needed.
%
% This template includes a few options for different layouts and
% content for various journals. Please consult a previous issue of
% your journal as needed.
%
%%%%%%%%%%%%%%%%%%%%%%%%%%%%%%%%%%%%%%%%%%%%%%%%%%%%%%%%%%%%%%%%%%%
%
%
\RequirePackage{fix-cm}
\documentclass[twocolumn]{svjour3}          % twocolumn

\usepackage{polski}
\usepackage[polish]{babel}
\usepackage[utf8]{inputenc}

\smartqed  % flush right qed marks, e.g. at end of proof

\usepackage{graphicx}
\usepackage{url}

\usepackage{tikz}
\usepackage{pgfplots}

\begin{document}
% -----------------------------------------------------------------------

\title{Analiza literatury Big Data}
%\subtitle{Do you have a subtitle?\\ If so, write it here}

\author{Grzegorz Cichosz \and
    Michał Lenart \\
    Mateusz Rzeszutek \and
    Dariusz Świętek
}

\authorrunning{Cichosz, Lenart, Rzeszutek, Świętek} % if too long for running head

\date{\today}

% -----------------------------------------------------------------------
\maketitle

\begin{abstract}
    Big data oznacza dosłownie ,,duże dane''. Termin ten powstał stosunkowo niedawno i jest on używany wtedy, gdy mówi się o przetwarzaniu bardzo dużych ilości danych. Na zaistnienie tej dość specyficznej dziedziny danych miał duży wpływ rozwój Internetu -- ilość danych pochodzących chociażby z Twittera jets monstrualna.
    \keywords{Big Data \and bazy danych \and przetwarzanie danych \and Internet}
\end{abstract}

\section{Wstęp}
\label{sec:wstep}

% TODO

% section wstep (end)

% --- IMAGE & TABLE EXAMPLES ---
% For one-column wide figures use
\begin{figure}
    % Use the relevant command to insert your figure file.
    % For example, with the graphicx package use
    \includegraphics{example.eps}
    % figure caption is below the figure
    \caption{Please write your figure caption here}
    \label{fig:1}       % Give a unique label
\end{figure}
%
% For two-column wide figures use
\begin{figure*}
    % Use the relevant command to insert your figure file.
    % For example, with the graphicx package use
    \includegraphics[width=0.75\textwidth]{example.eps}
    % figure caption is below the figure
    \caption{Please write your figure caption here}
    \label{fig:2}       % Give a unique label
\end{figure*}
%
% For tables use
\begin{table}
    % table caption is above the table
    \caption{Please write your table caption here}
    \label{tab:1}       % Give a unique label
    % For LaTeX tables use
    \begin{tabular}{lll}
        \hline\noalign{\smallskip}
        first & second & third  \\
        \noalign{\smallskip}\hline\noalign{\smallskip}
        number & number & number \\
        number & number & number \\
        \noalign{\smallskip}\hline
    \end{tabular}
\end{table}

% -----------------------------------------------------------------------
\nocite{*}
\bibliographystyle{spphys}
\bibliography{literatura}

\end{document}
% end of file template.tex

