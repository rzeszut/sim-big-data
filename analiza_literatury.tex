\section{Analiza literatury}
\label{sec:analiza_literatury}

\subsection{Metody}
\label{sub:metody}
Big Data wymaga wyjątkowych technologii do efektywnego przetwarzania olbrzymich ilości danych w rozsądnym czasie. W 2011 roku firma {McKinsley} w swoim raporcie \cite{McKinsey2011} zaproponowała użycie do analizy Big Data takich technik jak testy A/B, crowdsourcing, eksploracja danych, algorytmy genetyczne etc.

\subsubsection{A/B testing}
\label{sub:a/b_testing}
Testy A/B jest to metoda badawcza służąca do wybrania lepszego rozwiązania. W założeniu osoba przeprowadzająca test posiada dwie wersje danego elementu i metrykę określającą optymalność. Aby wybrać lepsze rozwiązanie, obie wersje poddawane są temu samemu eksperymentowi. Na końcu mierzony jest wcześniej ustalony wskaźnik jakości i wybierane jest lepsze rozwiązanie \cite{paras10}.

\subsubsection{Association rule learning}
\label{sub:association_rule_learning}
Association rule learning jest to popularna i dobrze zbadana metoda szukania relacji pomiędzy zmiennymi w dużych bazach danych \cite{tan2005introduction}.

\subsubsection{Crowdsourcing}
\label{sub:crowdsourcing}
Termin ''crowdsourcing'' został po raz pierwszy zdefiniowany i użyty przez dziennikarza magazynu Wired Jeffa Howe’a \cite{Howe2006}. Odnosi się on do procesu, w którym wszelkie potrzebne moduły, serwisy, pomysły czy materiały pozyskiwane są od dużej grupy ludzi (zwłaszcza od społeczności online) w formie outsourcingu. Jest to metoda zastępująca tradycyjnych pracowników bądź dostawców. Proces ten jest często używany do podziału żmudnej pracy lub w celu pozyskiwania funduszy na finansowanie nowych firm i organizacji charytatywnych \cite{Howe2006}.

Crowdsourcing łączy wysiłki wielu wolontariuszy i pracowników tymczasowych. Każdy z uczestników wnosi do projektu niewielki nakład pracy, co przy dużej liczbie ochotników powoduje osiągnięcie określonego celu.

Crowdsourcing różni się od outsourcingu tym, że wolontariuszem może zostać każdy, a nie tylko wybrana grupa ludzi.

\subsubsection{Eksploracja danych}
\label{sub:eksploracja_danych}
% TODO

\subsubsection{Data Integration}
\label{sub:data_integration}
% TODO

\subsubsection{Algorytmy genetyczne}
\label{sub:algorytmy_genetyczne}
% TODO

\subsubsection{Uczenie maszynowe}
\label{sub:uczenie_maszynowe}
% TODO
Uczenie maszynowe to dziedzina informatyki mocno powiązana ze sztuczną inteligencją. Uczenie maszynowe zajmuje się projektowaniem i konstrukcją algorytmów i systemów, których potrafią się ucyzć (zdobywać wiedzę) z uzyskanych danych.

\subsubsection{NLP}
\label{sub:nlp}
% TODO

\subsubsection{Przetwarzanie sygnałów}
\label{sub:przetwarzanie_sygnalow}
% TODO

\subsubsection{Symulacja}
\label{sub:symulacja}
% TODO

\subsubsection{Szereg czasowy}
\label{sub:szereg_czasowy}
% TODO

\subsubsection{Wizualizacja}
\label{sub:wizualizacja}
% TODO

\subsection{Implementacje}
% TODO

\subsection{Zastosowania implementacji}
%TODO

%TODO trendy itp.

% section Analiza literatury (end)
