\section{Analiza literatury}
\label{sec:analiza_literatury}

\subsection{Metody}
\label{sub:metody}
Big Data wymaga wyjątkowych technologii do efektywnego przetwarzania olbrzymich ilości danych w rozsądnym czasie. W 2011 roku firma {McKinsley} w swoim raporcie \cite{McKinsey2011} zaproponowała użycie do analizy Big Data takich technik jak testy A/B, crowdsourcing, eksploracja danych, algorytmy genetyczne etc.

\subsubsection{A/B testing}
\label{sub:a/b_testing}
Testy A/B jest to metoda badawcza służąca do wybrania lepszego rozwiązania. W założeniu osoba przeprowadzająca test posiada dwie wersje danego elementu i metrykę określającą optymalność. Aby wybrać lepsze rozwiązanie, obie wersje poddawane są temu samemu eksperymentowi. Na końcu mierzony jest wcześniej ustalony wskaźnik jakości i wybierane jest lepsze rozwiązanie \cite{paras10}.

\subsubsection{Association rule learning}
\label{sub:association_rule_learning}
Association rule learning jest to popularna i dobrze zbadana metoda szukania relacji pomiędzy zmiennymi w dużych bazach danych \cite{tan2005introduction}.

\subsubsection{Crowdsourcing}
\label{sub:crowdsourcing}
Termin ''crowdsourcing'' został po raz pierwszy zdefiniowany i użyty przez dziennikarza magazynu Wired Jeffa Howe’a \cite{Howe2006}. Odnosi się on do procesu, w którym wszelkie potrzebne moduły, serwisy, pomysły czy materiały pozyskiwane są od dużej grupy ludzi (zwłaszcza od społeczności online) w formie outsourcingu. Jest to metoda zastępująca tradycyjnych pracowników bądź dostawców. Proces ten jest często używany do podziału żmudnej pracy lub w celu pozyskiwania funduszy na finansowanie nowych firm i organizacji charytatywnych \cite{Howe2006}.

Crowdsourcing łączy wysiłki wielu wolontariuszy i pracowników tymczasowych. Każdy z uczestników wnosi do projektu niewielki nakład pracy, co przy dużej liczbie ochotników powoduje osiągnięcie określonego celu.

Crowdsourcing różni się od outsourcingu tym, że wolontariuszem może zostać każdy, a nie tylko wybrana grupa ludzi.

\subsubsection{Eksploracja danych}
\label{sub:eksploracja_danych}
% TODO

\subsubsection{Data Integration}
\label{sub:data_integration}
% TODO

\subsubsection{Algorytmy genetyczne}
\label{sub:algorytmy_genetyczne}
% TODO

\subsubsection{Uczenie maszynowe}
\label{sub:uczenie_maszynowe}
% TODO
Uczenie maszynowe to dziedzina informatyki mocno powiązana ze sztuczną inteligencją. Uczenie maszynowe zajmuje się projektowaniem i konstrukcją algorytmów i systemów, których potrafią się ucyzć (zdobywać wiedzę) z uzyskanych danych.

Algorytmy uczenia maszynowego można podzielić na dwa rodzaje: nadzorowane i nienadzorowane. Algorytmy nadzorowane na wejściu dostają wzorcowy zbiór danych, który służy im za bazę do oceny bądź klasyfikacji nowych danych. Algorytmy nienadzorowane nie mają takiego zestawu danych. Przykładem algorytmu nienadzorowanego jest algorytm klastrowania \textit{k-means} \cite{sugar03}.

\subsubsection{Przetwarzanie języków naturalnych}
\label{sub:nlp}
Przetwarzanie języków naturalnych (\textit{NLP -- Natural Language Processing}) jest przykładowym wykorzystaniem uczenia maszynowego i wiedzy z zakresu lingwistyki. Jest to zbiór technik służących do analizy języków naturalnych -- czyli takich, jakimi posługują się ludzie na co dzień. Jednym z bardziej znanych zastosowań tych algorytmów jest analiza sentymentów postów na Twitterze \cite{agarwal11}. Dane pochodzące z Twittera były też używane do estymowania aktywności wirusa świńskiej grypy H1N1 \cite{signorini11}.

\subsubsection{MapReduce}
\label{sub:mapreduce}
MapReduce jest to platforma stworzona przez pracowników firmy Google.  W 2004 r. dwóch z nich (Jeffrey Dean oraz Sanjay Ghemawat) wydali artykuł \cite{Dean08}, w którym przedstawili nowy system. 

System ten ma za zadanie tworzenie aplikacji działających jednocześnie na tysiącach komputerów. Jak nazwa wskazuje, mamy w nim doczynienia z dwoma częściami obliczeń, a mianowicie mapowaniem (map) oraz redukcją (reduce).

Największym plusem tego podejścia jest możliwość łatwego rozdzielenia operacji na różne serwery, ponieważ można założyć, że każda operacja mapowania jest niezależna od pozostałych.


\subsection{Statystyka}
\label{sub:statystyka}
Statystyka jest nauką matematyczną zajmującą się zbieranem, organizjacją, analizą i interpretacją, a także prezentacją danych \cite{statystyka:podrecznik}. Metody statystyczne są często używane do znajdywania i określania związków w zestawach danych. Przykładem metody statystycznej są testy A/B opisane w sekcji \ref{sub:a/b_testing}.

\subsubsection{Przetwarzanie sygnałów}
\label{sub:przetwarzanie_sygnalow}
% TODO

\subsubsection{Symulacja}
\label{sub:symulacja}
% TODO

\subsubsection{Szereg czasowy}
\label{sub:szereg_czasowy}
% TODO

\subsubsection{Wizualizacja}
\label{sub:wizualizacja}
Techniki wizualizacji naukowej są używane do tworzenia obrazków, diagramów, wykresów bądź animacji, których zadaniem jest ułatwienie zrozumienia i przekzazanie pewnych wniosków wynikających z analizy danych \cite{lawrence1994}. Do tworzenia prostych wykresów i diagramów często używa się takiego oprogramowania jak R \cite{www:R} bądź Octave \cite{www:octave}.

\subsection{Implementacje}
% TODO

\subsection{Zastosowania implementacji}
%TODO

%TODO trendy itp.

% section Analiza literatury (end)
